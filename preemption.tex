\documentclass[11pt, a4paper]{article}
% \documentclass[11pt, a4paper]{scrartcl}

\usepackage[a4paper,lmargin={3cm},rmargin={3.5cm}, tmargin={2.5cm},bmargin = {2.5cm}]{geometry}
\usepackage{setspace}
\usepackage{indentfirst}
% \onehalfspacing
\usepackage{enumitem}
\usepackage{amsmath, amssymb}
\usepackage{graphicx}

\newcommand{\mh}[1]{\noindent\emph{#1}}
\newcommand{\Ssa}{S_{\sigma\alpha}}
\newcommand{\sa}{\sigma\alpha}
\newcommand\given[1][]{\:#1\vert\:}

\renewcommand{\i}[1]{\emph{#1}}

\usepackage[backend=biber, authordate, ibidtracker=context]{biblatex-chicago}
\addbibresource{preemption.bib}

\title{\textbf{How to Deal With Expert Testimony?} \\Preemption View and Total Evidence View tested in the Olsson and Vallinder Model for Social Networks }
\author{Class Paper: Formal Methods II \\ MCMP @ LMU Munich \\ Conrad Friedrich \\ \texttt{conradfriedrich@posteo.net}}

\begin{document}

\maketitle
\abstract{}
\section{Introduction}
How to deal with expert testimony? Mention Goldman, Elga etc.

Show position by Kelly
Show position of Constantin and Grundmann

Show need to make this precise. In the case of preemption view, how to model? 
kelly's TEV for credences is just the same as bayesianism 

How to evaluate the rational strategy formally?
Epistemic Value!

\section{Model Description}

The Model used in this paper follows the one proposed by \textcite{Olsson2013} rather closely. I adopt most of their formalizations and assumptions. To enable the reader to make use of the derivations in \textcite{Angere2010}, I adopt their notations as well. I summarize the model, describe how I use it for the present purposes and make note of any deviation.

The core entities of the model are \i{agents} and their properties. Agents can be connected to one another, such that they form a network. Since these connections enable communication and the agents are meant to represent people, this structure can be seen as a social network.

The model has certain starting parameters, which I describe below, and discrete timesteps. Each of these enable the update of properties by the rules specified further below.

\subsection{Starting Parameters}

To keep things manageable, there is a single \i{true} target propostion $p$ which the agents have a credence $C^t_\alpha(p)$ towards. Each timestep, agents can receive information from a \i{source}. These can be (i) other agents in the social network or (ii) their own inquiry. The objective chance that an agents inquires and gets it right is called that agent $\alpha$'s \i{aptitude}: $P(S_{\iota \alpha}p \given S_{\iota \alpha} \land p)$, where $S_{\iota \alpha}$ is the chance that agent $\alpha$ inquires for evidence, regardless of the results, and $S_{\iota \alpha} p$ the chance that she inquires and her inquiry yields $p$ as a result. The chance that she inquires at all, $P(S_{\iota \alpha})$, is called her \i{activity}. For the sake of simplicity, these chances are modeled as time invariant.

Sources can be more or less reliable. The reliability is assumed to be symmetric.
\[ 
R_{\sigma \alpha} =_{df.} P(\Ssa p \given \Ssa \land p) = P(\Ssa \neg p \given \Ssa \land \neg p)
\]
This is, of course, just the agent's aptitude. 

Each agent trusts each source to a certain extent. This is expressed in the agent's credence in the reliability of the source:
\[ 
    C^t_{\alpha}(a \leqslant R_{\sa} \leqslant b) = \int_a^b \tau^t_{\sa}(\rho) d\rho
\]
where $\tau^t_{\sa}: [0,1] \rightarrow \mathbb{R}^+$ is a probability density function such that 
\[
    C^t_{\alpha}(0 \leqslant R_{\sa} \leqslant 1) =  \int_0^1 \tau^t_{\sa} (\rho) d(\rho) = 1
\]which is a very plausible requirement on a rational credence function.\footnote{Add Stuff here about the Olsson and Vallinder Model and how their's is slightly different.}

\begin{description}
    \item[Example.] An agent $\alpha$ with a healthy trust in her own inquisitive abilities and who is not easily swayed in this trust may have a trust function $\tau^t_{\iota\alpha}$ described by a beta distribution
\[
    \tau^t_{\iota\alpha} (\rho) = \frac{1}{B(a,b)} \rho^{a - 1} {(1 - \rho)}^{b-1}
\]
with normalising beta function $B(a,b)$ and parameters $a, b$ chosen such that $\tau^t_{\iota\alpha}$ is densest around whatever a `healthy trust' amounts to, let's say 0.85. (see Fig.\ref{fig1}). Interestingly, the shape of the curve correlates with how resilient the funtion behaves upon updating. More on that below.
\end{description}

\begin{figure}[ht]
	\centering
    \includegraphics[width=0.5\textwidth]{Figure_1.png}
	\caption{Example Trust function}
    \label{fig1}
\end{figure}

Apart from inquiries, a source can also be another agent, through testimony. An agent $\alpha$ therefore trusts herself at time $t$ with $\tau^t_{\iota\alpha}$ and another agent $\beta$ with $\tau^t_{\beta\alpha}$.

The exchange of information and the evolution of the trust functions and the credences in $p$ are the most important going-ons in this model, which I'll summarize next.

\subsection{Inquiry and Communication} 
The passing of time is modeled in discrete time steps. At each time step, each agent updates her credence and her trust function. The new information which motivates these changes are messages from the sources, which the agent receives if
\begin{enumerate}[label = (\roman*)]
    \item the agent inquires herself with probability $P(S_{\iota\alpha})$. The result of the inquiry is determined by the agent's aptitude. Or,
    \item another agent $\beta$ is connected to agent $\alpha$ through the social network and decides to share her opinion. This is modeled, in slight deviation from the Olsson-Vallinder model, by the following conditions: 
        \begin{enumerate}[label = (\alph*)]
            \item Agent $\beta$ is sufficiently confident in $p$ resp. $\neg p$, determined by some faily high threshold value for her credence, and
            \item agent $\beta$ received new information herself in the same time step. In the current implementation, this just is only fulfilled by own inquiry.  
        \end{enumerate}
\end{enumerate}

EXPLAIN why only when new information!!

\section{}

Model of a social network. Agents with links. Agents have beliefs that are updated over time.

Agent's properties
Communication. Diverges from Constantin and Grundmann
Reliability
Updating etac\
esp. \ the time step issue. \ can repeated statements of the authority really be independent of the ones before?
Epistemic Value
CENTRAL ASSUMPTIONS

\section{Experiment}
How to setup a specific experiment? Expert Layperson structure, how to set up credences, metadistributions, 

\section{Experiment: Results}

\nocite{*}
\printbibliography{}
\end{document}

